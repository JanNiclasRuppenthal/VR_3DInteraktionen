\documentclass{beamer}

\mode<presentation>{\usetheme{Madrid}}
\usepackage{graphicx}
\usepackage{multimedia}
\usepackage{hyperref}

\usepackage[utf8]{inputenc}
\usepackage[ngerman]{babel}
\usepackage{amsmath, amssymb, amsthm}

\usepackage{subcaption}
\usepackage[T1]{fontenc}
%\usepackage[sort&compress]{natbib}

\usepackage{lmodern}
\usepackage{caption}


\author[Jan Niclas Ruppenthal, Michael Feldmann, Philipp Geier]{}
\title[]{1. Übung zur Vorlesung
Virtual Reality\\ Ames-Raum}
\institute[Universität Trier]{}
\date[06. Mai 2024]{}
\beamertemplatenavigationsymbolsempty
\setbeamertemplate{footline}
{
  \leavevmode%
  \hbox{%
  \begin{beamercolorbox}[wd=.50\paperwidth,ht=2.25ex,dp=1ex,center]{author in head/foot}%
    \usebeamerfont{author in head/foot}\insertshortauthor%~~\beamer@ifempty{\insertshortinstitute}{}{(\insertshortinstitute)}
  \end{beamercolorbox}%
  \begin{beamercolorbox}[wd=.50\paperwidth,ht=2.25ex,dp=1ex,right]{date in head/foot}%
    \usebeamerfont{date in head/foot}\insertshortdate{}\hspace*{2em}
    \insertframenumber{} / \inserttotalframenumber\hspace*{2ex} 
  \end{beamercolorbox}}%
  \vskip0pt%
}
%—-------------------------------------------------------------

\begin{document}
{
  \usebackgroundtemplate{\includegraphics[width=1.2\paperwidth]{unitrier}}
  \begin{frame}
    \maketitle
  \end{frame}
}
    
    %\begin{frame}
       % \frametitle{Inhalt}
		%\tableofcontents
	%\end{frame}
	
%—------------------------------------------------------


\begin{frame}{Aufgabenteil (a)}
\begin{itemize}
\item Erstellen eines Ames-Raums in Unity.
\item Verstärkung der Illusion durch die Bewegung der Charaktere.
\item Auflösung der Illusion durch eine Kamerafahrt.
\end{itemize}
\end{frame}

\begin{frame}{Kamerafahrt (Version 1)}
\begin{figure}
    \centering
    \movie[externalviewer]{
\includegraphics[width=0.9\textwidth, keepaspectratio]{ezgif-frame-001}
}{MovementAmesRoom.mp4}
\caption{Kamerafahrt im Unity Editor ohne Wand}
\end{figure}
\end{frame}

\begin{frame}{Kamerafahrt (Version 2)}
\begin{figure}
    \centering
    \movie[externalviewer]{
\includegraphics[width=0.9\textwidth, keepaspectratio]{ezgif-frame-001}
}{MovementAmesRoom.mp4}
\caption{Kamerafahrt im Unity Editor mit Wand}
\end{figure}
\end{frame}


\begin{frame}{Aufbau der vorderen Wand}
\begin{figure}
    \centering
\includegraphics[width=0.9\textwidth, keepaspectratio]{geogebra-export_Vorderseite}
\caption{Maße der vorderen Wand}
\end{figure}
\end{frame}


\begin{frame}{Aufbau des Bodens}
\begin{figure}
    \centering
\includegraphics[width=0.9\textwidth, keepaspectratio]{geogebra-export_Boden}
\caption{Maße des Bodens}
\end{figure}
\end{frame}



\begin{frame}{Hinzufügen der Bodentextur}
\begin{figure}
    \centering
\includegraphics[width=0.75\textheight, keepaspectratio]{nick}
\caption{Checkerboard Boden}
\end{figure}
\end{frame}



\begin{frame}{Prototyp}
\begin{figure}
    \centering
\includegraphics[width=0.9\textwidth, keepaspectratio]{prototyp}
\caption{Ames-Raum Prototyp}
\end{figure}
\end{frame}


\begin{frame}{Thematisierung}
\begin{figure}
    \centering
\includegraphics[width=0.9\textwidth, keepaspectratio]{astro}
\caption{Astronaut Asset}
\end{figure}
\end{frame}


\begin{frame}{Thematisierung}
\begin{itemize}
\item Ersetzen der Beans mit dem Astronaut Asset.
\end{itemize}
\begin{figure}
    \centering
\includegraphics[width=0.9\textwidth, keepaspectratio]{thema1}
\caption{Astronaut Asset}
\end{figure}
\end{frame}


\begin{frame}{Thematisierung}
\begin{itemize}
\item Ersetzen der Prototyp Bilder mit Planeten und Raketen.
\end{itemize}
\begin{figure}
    \centering
\includegraphics[width=0.9\textwidth, keepaspectratio]{thema2}
\caption{Bilder}
\end{figure}
\end{frame}

\begin{frame}{Thematisierung}
\begin{itemize}
\item Anpassung der Farben an das Thema.
\end{itemize}
\begin{figure}
    \centering
\includegraphics[width=0.9\textwidth, keepaspectratio]{thema3}
\caption{Farbgebung}
\end{figure}
\end{frame}

\begin{frame}{Thematisierung}
\begin{itemize}
\item Hinzufügen eines sich bewegenden Sternenhimmels.
\end{itemize}
\begin{figure}
    \centering
\includegraphics[width=0.9\textwidth, keepaspectratio]{thema4}
\caption{Sternenhimmel}
\end{figure}
\end{frame}

\begin{frame}{Hinzufügen von Skripten}
\begin{itemize}
\item Für die Astronauten.
\item Für die Kamerafahrt.
\item Für den Sternenhimmel.
\end{itemize}
\end{frame}

\begin{frame}{Aufgabenteil (b)}
\begin{itemize}
\item Illusion in VR erzeugen.
\item Illusion durch Bewegung auflösen.
\item Illusion durch Werfen von Objekten auflösen.
\end{itemize}
\end{frame}


\begin{frame}{VR Illusion}
    \centering
    \movie[externalviewer]{
\includegraphics[width=0.9\textwidth, keepaspectratio]{ezgif-frame-001}
}{MovementAmesRoom.mp4}
\end{frame}


\begin{frame}{Werfen}
    \centering
    \movie[externalviewer]{
\includegraphics[width=0.9\textwidth, keepaspectratio]{ezgif-frame-001}
}{MovementAmesRoom.mp4}
\end{frame}


	
    	
    	
    	
\end{document}
